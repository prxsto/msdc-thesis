\documentclass [11pt, proquest] {uwthesis}[2020/02/24]
%\usepackage{subfiles}
\setcounter{tocdepth}{1}  % Print the chapter and sections to the toc
 
\begin{document}
 
% === preliminary pages

\prelimpages
 
% --- title and copyright pages

\Title{Machine Learning Energy Use Intensity Optimization of Accessory Dwelling Units}
\Author{P. Arthur Pape}
\Year{2022}
\Program{Department of Architecture}
\Degree{Master of Science in Architecture (Design Computing)}
\Degreetext{A thesis submitted in partial \\
fulfillment of the requirements\\
for the degree of}
\textofCommittee{Committee}
\Chair{Rick Mohler}{Associate Professor}{Department of Architecture}
\Chair{Tomas Mendez Echenagucia}{Assistant Professor}{Department of Architecture}

\titlepage  

\copyrightpage

% --- abstract
\setcounter{page}{-1}
\abstract{
\textbf{To be completed}
}

 
% --- contents & etc.
\tableofcontents
\listoffigures
%\listoftables  % I have no tables

% --- glossary
%\include{./chapters/glossary}

% --- acknowledgments
%\acknowledgments{
  % \vskip2pc
  % {\narrower\noindent
\begin{bf} WIP	
\end{bf}


  % \par}
}

% === chapters

\textpages
%\chapter {Introduction}

\section{Context}

The unprecedented housing crisis in which the United States is currently facing requires both updated, progressive policy and novel design ideas. Stabilizing property values is socially beneficial to all residents \cite{10.2307/24392672}. Hence, a twin focus on equity and sustainability is required for an effective solution. This predicament the housing market finds itself in can largely be traced to zoning regulation. Single family housing zones across the country, as they currently exist, restrict the density necessary to house all citizens at just prices. Resultantly, rising housing costs, whether rental or mortgage, stem from a decreased supply of housing. For many, housing costs account for a greater share of their income, restricting their ability to afford basic necessities and decreasing overall quality of life. Due to the commodification of the housing market, increasing amounts of residents are becoming homeless and unable to find a foothold back into permanent housing \cite{routhierStateHomeless20212021}. As the realities of climate change begin to set in, the method employed to ease the housing crisis must also address sustainability. Further, people of color and those with lower levels of income are disproportionately affected by skyrocketing housing prices, and are also those most at risk of the enlarged rate of natural disasters due to our changing climate. Low income residents currently encounter asymmetric energy costs when compared to wealthier citizens, widening the disparity in the cost of living \cite{kontokostaEnergyCostBurdens2020a}. 

Traditional means of increasing housing stock is by increasing density and constructing multifamily housing structures. However, since the 1920’s, the trend has been larger plots of land per owner, leading to urban sprawl \cite{hertzImmaculateConceptionTheory15}. Proposed changes to land use policy is frequently met with pushback from homeowners due to lesser demand equaling a reduced property value. Ideally, single-family residential zones close to higher-density zoning would gradually be upzoned, or converted into multifamily residential or mixed, multifamily residential and commercial. As this is not currently feasible, cities around the country have begun to experiment and implement new regulations allowing for the construction of accessory dwelling units (ADUs). Accessory dwelling units are separate structures residing within existing single-family residential lots, typically used as rental housing, office space, or for housing an aging loved one close to home. ADUs fall into two sub-categories: attached and detached, with the former being separate, freestanding structures, while the latter is an expansion onto the existing structure. This thesis will be focusing solely on detached accessory dwelling units, or DADUs.

Due to their size, ADUs allow individuals to downsize the amount of space they occupy, reducing energy use and carbon footprint per capita. ADUs being a comparatively affordable typology of investment, combined with their intrinsic ability to increase property value, has led the way for new policy to allow for their construction \cite{chappleJumpstartingMarketAccessory2017}. Beyond restrictive zoning and permitting regulations, price contributes to the rate of accessory dwelling unit construction. In fact, recent 2019 survey data collected by the City of Seattle shows that homeowners desire an increased focus on sustainability and cost. Discovering an effective method for a reduction in energy use will increase construction rate of ADUs, benefitting both owners and occupants, while addressing climate change and housing injustices. It is of utmost importance for architects, policymakers, and residents to recognize that energy use and housing availability are not solely problems of climate or equity, respectively.\\

Traditional methods for evaluating energy performance of a proposed structure involve what is referred to as the modeling approach. This method requires an accurate 3D model to be constructed with specific materials applied, along with geographic information. The model is then simulated over the course of one calendar year to evaluate energy performance metrics, including energy use intensity (EUI). EUI is defined as the total energy consumed by a building over the course of one year, divided by gross square footage of the building (\(\frac{kW}{m^2\cdot yr}\)). This metric is helpful to compare energy use between buildings of varying sizes and typologies. EnergyPlus, an open-source building energy simulation program, is the most widely-used framework for this method in the United States. While precise, the modeling approach is time consuming on the scale of an individual design project. In contrast, the statistical approach is gaining popularity. These methods for calculating building energy performance rely on statistical inference using specified design inputs. This thesis investigates the use of surrogate modeling as an improved method to derive EUI. Surrogate modeling uses statistics and computation to estimate values to a high degree of accuracy. Using machine learning models to calculate EUI saves time for designers, but requires extra upfront resources to train.

\section{Previous work}

Efforts to increase the production of ADUs across Seattle have varied in goal and scope.  Beginning in 1994, Seattle began to allow ADUs only in single-family zones due to requirements in the Washington Housing Policy Act \cite{levyADUsPoliciesRacial2019}. However, the limitation at the time was that only AADUs were allowed. Following up on this, Seattle expanded the ADU rules to allow for DADUs in 2010. As only 50 out of 100,000 eligible properties constructed ADUs since this change, Seattle began exploring policies to make accessory dwelling units more accessible in 2014. In the process, the city conducted a Racial Equity Toolkit (RET) in 2018 to determine if proposed policy changes could reduce racial disparities in housing \cite{welchAccessoryDwellingUnits2021}. This report discovered that removing barriers to ADUs was beneficial to both affordability and displacement through increasing the choice involved in housing. Yet, it was found that removal of regulatory barriers was inadequate alone, resulting in the expansion of the Home Repair Loan Program. This modification allowed for greater access of funds to low income families to construct additional housing for either family or community members on existing properties \cite{welchAccessoryDwellingUnits2021}.

In 2019, an initiative to inform potential ADU owners named ADUniverse was created as a joint project between Seattle Office of Planning and Community Development (OPCD) and the Data Science for Social Good Program at the University of Washington \cite{mohlerADUniverseToolEScience2019}. ADUniverse’s goal was to understand where ADUs have historically been built in Seattle, and to identify potential issues with eligible lots, while attempting to estimate costs associated. As part of this project, a survey was conducted later in September of 2019 to begin to understand the design criteria most important to potential owners. Results showed that “low cost” was the criteria in first place with 48\% of those surveyed responding “very important”, followed by ‘green building’ at 46\%. Other more-specific priorities suggested include: longer-term environmental costs, site specific considerations, and predictability in both construction and cost \cite{PreapprovedPlansAccessory2019}. The results of this survey were then used to inform design submissions from firms, with ten of these being selected later by the City of Seattle to be featured on their website as pre-approved plans.

While these pre-approved designs are from many renowned firms who do indeed focus on sustainability, they will fall short on energy efficiency. Due to the site-agnostic nature of pre-approved plans, unique characteristics of each site will have influences on their effectiveness: shading from surrounding objects and structures, orientation to existing buildings, solar gain, etc. On the other hand, being pre-approved, many headaches regarding permitting and construction can be bypassed. To fully remedy this situation a modular approach would be recommended. In the meantime, the collective improvements to access and modifications to ADU rules has proven successful.

Following the new City of Seattle rules implemented in 2019 regarding ADUs and the collection of 10 pre-approved DADU designs in 2020 as part of ADUniverse, ADU production has increased immensely. As this comes during a global pandemic, the values would likely be larger under normal circumstances. According to OPCD, in 2020 the number of AADU and DADU permits increased 53\% and 112\% respectively, over similar numbers from 2019 \cite{welchAccessoryDwellingUnits2021}. With this increase in production comes a need to continue to maintain this boost in production while simultaneously optimizing the energy use of these newly permitted units.



\section{Research questions}

As ADU production ramps up dramatically in Seattle, and across the country, there must be an emphasis on the sustainability of each new construction. 

\bf{Main research questions}
\begin{itemize}
	\item Can a predictive design tool be leveraged increase production of DADUs in Seattle while reducing energy and carbon intensity of new units?
	\item Can design results generated by the tool verify city planning policy initiatives?
\end{itemize}

\bf{Secondary questions}
\begin{itemize}
	\item What effects on EUI does sharing walls (removing rear setbacks) and introducing a floor-area ratio benefit for retaining existing structures have?
	\item How does the energy performance of the 10 pre-approved DADU designs compare to a site-specific DADU design?
	\item What is the effect of the design tool empowering potential DADU owners to see energy quantities beforehand?
	\item Which design constraints impact DADU energy use the greatest?
\end{itemize}


\subsection{Aims and objectives}

Providing designers with real-time energy and carbon consequences of design decisions would greatly increase sustainability efforts across the industry. 

tool useful within existing code even if results dont push foward policy
 
1. provide designers with realtime energy and therefore carbon consequences of design decisions
2. how results can prove policy

\section{Research methods and outline}

%TODO 
\textbf{[WIP] Write miniature methods section and detail the outline of the thesis}

%\include{./chapters/02_background}
%\include{./chapters/03_literature-review}
%\include{./chapters/04_methodology}
%\include{./chapters/05_results}
%\include{./chapters/06_conclusions-discussion}

% === bibliography



% === appendices



% ========== Chapter 2
 
\chapter{A Brief \\ Description of \protect\TeX}
 
The \TeX\ formatting program is the creation of
Donald Knuth of Stanford University.
It has been implemented on nearly every general purpose computer and
produces exactly the same copy on all machines.
 
\section{What is it; why is it spelled that way; 
and what do
really long section titles look like in the text and in the
Table of Contents?}
 
\TeX\ is a formatter.  A document's format is controlled
by commands embedded in the text.  
\LaTeX\ is a special version of \TeX---preloaded
with a voluminous set of macros that simplify most
formatting tasks.
 
\TeX\ uses {\it control sequences} to control
the formatting of a document.  These control sequences are usually
words or groups of letters prefaced with the backslash character
({\tt\char'134}).
For example,
Figure \ref{start-2} shows the text that printed the beginning
of this chapter.  Note the control sequence \verb"\chapter" that
instructed \TeX\ to start a new chapter, print the title, and
make an entry in the table of contents.  It is an example
of a macro defined by the \LaTeX\ macro package.
The control sequence \verb"\TeX", which prints the word \TeX,
is a standard macro from the {\it\TeX book}.
The short control sequence \verb"\\" in the title instructed \TeX\ to
break the title line at that point.
This capability is an example of an extension to \LaTeX\
provided by the uwthesis document class.

Most of the time \TeX\ is simply building paragraphs from
text in your source files.  No control sequences are involved.
New paragraphs are indicated by a blank line in the
input file.
Hyphenation is performed automatically.
 
\section{\TeX books}
 
The primary reference for \LaTeX\ is Lamport's second edition
of the \textit{\LaTeX\ User's Guide}\cite{Lbook}.
It is easily read and should be sufficient for thesis formatting.
See also the \textsl{\LaTeX\ Companion}\cite{companion} for descriptions
of many add-on macro packages.

Although unnecessary for thesis writers, the \textsl{\TeX book}
is the primary reference for \TeX sperts worldwide.
 
\section{Mathematics}
 
The thesis class does not expand on \TeX's
or \LaTeX's
comprehensive treatment of mathematical equation printing.%
\label{c2note}\footnote{%
% a long footnote indeed.
 Although many \TeX-formatted documents contain no
 mathematics except the page numbers, it seems appropriate
 that this paper, which is in some sense about \TeX,
 ought to demonstrate an equation or two.  Here then, is a statement 
 of the {\it Nonsense Theorem}.
 
 \smallskip
 \def\RR{{\cal R\kern-.15em R}}
 {\narrower\hangindent\parindent Assume a universe $E$ and a symmetric function
  $\$$ defined on $E$, such that for each $\$^{yy}$ there exists a
  $\$^{\overline{yy}}$, where $\$^{yy} = \$^{\overline{yy}}$.
  For each element $i$ of $E$ define
  ${\cal S}(i)=\sum_i \$^{yy}+\$^{\overline{yy}}+0$.
  Then if $\RR$ is that subset of $E$ where $1+1=3$,
  for each $i$
  $$\lim_{\$\to\infty}\int {\cal S}di =
      \cases{0,&if $i\not\in\RR$;\cr
             \infty,&if $i\in\RR$.\cr}$$
  \par}} % end of the footnote
%
The {\it\TeX book}\cite{book}, {\it \LaTeX\ User's Guide}\cite{Lbook},
and {\it The \LaTeX\ Companion}\cite{companion}
thoroughly cover this topic.
 
% ========== Chapter 3
 
\chapter{The Thesis Unformatted}
 
This chapter describes the uwthesis class (\texttt{uwthesis.cls},
version dated 2014/11/13)
in detail 
and shows how it was used to format the thesis.
A working knowledge of Lamport's \LaTeX\ manual\cite{Lbook} is assumed.
 
\section{The Control File}
 
The source to this sample thesis is a single file
only because ease of distribution was a concern.
You should not do this.  Your task will be much easier if you
break your thesis into several files:  a file for the preliminary pages,
a file for each chapter,  one for the glossary, and one for each
appendix.  Then use a control file to tie them all together.
This way you can edit and format parts of your thesis much more
efficiently.
 
Figure~\ref{control-file} shows a control file that
might have produced this thesis.
It sets the document style, with options and parameters,
and formats the various parts of the thesis---%
but contains no text of its own.
 
 
%  control file caption and figure
%
%
\begin{figure}[p]
 \begin{fullpage}
  \uwsinglespace
  \begin{verbatim}
    % LaTeX thesis control file
 
 
    \begin{document}
 
    % preliminary pages
    %
    \prelimpages
    \include{prelim}
 
    % text pages
    %
    \textpages
    \include{chap1}
    \include{chap2}
    \include{chap3}
    \include{chap4}
 
    % bibliography
    %
    \bibliographystyle{plain}
    \bibliography{thesis}
 
    % appendices
    %
    \appendix
    \include{appxa}
    \include{appxb}
 
    \include{vita} 
    \end{document}
  \end{verbatim}
  \caption[A thesis control file]%
   {\narrower A thesis control file ({\tt thesis.tex}).
   This file is the input to \LaTeX\ that will produce a
   thesis.  It contains no text, only commands which
   direct the formatting of the thesis.
   }
  \label{control-file}
 \end{fullpage}
\end{figure}
 
The first section, from the \verb"\documentclass" to
the \verb"\begin{document}", defines the document class and options.
This sample thesis specifies the \texttt{proquest} style, which is now
required by the Graduate School and is the default.  
Two other, now dated, other styles are available:  \verb"twoside", which is similar but 
produces a wider binding margin and is more suitable for paper printing; and
\verb"oneside", which is really old fashoned.
This sample also specified a font size
of 11 points. 
Possible font size options are: \verb"10pt", \verb"11pt", and \verb"12pt".
Default is 12 points, which is the preference
of the Graduate School. If you choose a smaller size be sure to
check with the Graduate School for acceptability.  The smaller fonts
can produce very small sub and superscripts.

Include most additional formatting packages with \verb"\usepackage",
as describe by Lamport\cite{Lbook}.  The one exception to this
rule is the \verb"natbib" package.  Include it with the \verb"natbib"
document option.
 
Use the \verb"\includeonly" command to format only a part of your
thesis.  See Lamport\cite[sec. 4.4]{Lbook} for usage and limitations.

 
\section{The Text Pages}
 
A chapter is a major division of the thesis.  Each chapter begins
on a new page and has a Table of Contents entry.
 
\subsection{Chapters, Sections, Subsections, and Appendices}
 
 
Within the chapter title use a \verb"\\" control sequence to separate lines
in the printed title (recall Figure \ref{start-2}.).
The \verb"\\" does not affect the Table of Contents entry.
 
Format appendices just like chapters.
The control sequence \verb"\appendix" instructs \LaTeX\ to
begin using the term `Appendix' rather than `Chapter'.
 
 
Specify sections and subsections of a chapter 
with  \verb"\section" and \verb"\subsection", respectively.
In this thesis chapter and section
titles are written to the table of contents.
Consult Lamport\cite[pg. 176]{Lbook} to see which
subdivisions of the thesis can be written to the table of contents.
The \verb"\\" control sequence is not permitted in section and
subsection titles.
 
 
\subsection{Footnotes}
 
\label{footnotes}
 Footnotes format as described in the \LaTeX\ book.  You can also
 ask for end-of-chapter or end-of-thesis notes.
 The thesis class will automatically set these up if
 you ask for the document class option \texttt{chapternotes}
 or \texttt{endnotes}.  
 
If selected, chapternotes will print automatically.  If you choose
endnotes however you must explicitly indicate when to print the notes 
with the command \verb"\printendnotes".  See the style guide for
suitable endnote placement.  

\subsection{Figures and Tables}
Standard \LaTeX\ figures and tables, see Lamport\cite[sec.~C.9]{Lbook},
normally provide the most convenient means to position the figure.
Full page floats and facing captions are exceptions to this rule.

If you want a figure or table to occupy a full page enclose the
contents in a \texttt{fullpage} environment.  
See figure~\ref{facing-caption}.

\subsubsection{Facing pages}
Facing page captions are an artifact of traditional, dead-tree printing,
where a left-side (even) page faces a right-side (odd) page.

In the \texttt{twoside} style, a facing caption
is full page caption for a full page figure or table
and should face the illustration to which it refers.
You must explicitly format both pages. 
The caption part appears on an even page
(left side) and the figure or table
comes on the following odd page (right side).
Enclose the float contents for the caption 
in a \texttt{leftfullpage} environment,
and enclose the float contents for the figure or table 
in a \texttt{fullpage} environment.
The first page (left side) contains the caption. The second page
(right side) could be left blank.  A picture or graph might be pasted onto
this space. See figure~\ref{facing-caption}.


\begin{figure}[t]
\uwsinglespace
\begin{verbatim}
     \begin{figure}[p]% the left side caption
       \begin{leftfullpage}
         \caption{ . . . }
       \end{leftfullpage}
     \end{figure}
     \begin{figure}[p]% the right side space
       \begin{fullpage}
          . . .
          ( note.. no caption here )
       \end{fullpage}
     \end{figure}
\end{verbatim}
\caption[Generating a facing caption page]{This text would create a
  double page figure in the two-side styles. }
\label{facing-caption}
\end{figure}
 
You can use these commands with the \texttt{proquest} style, but they have little
effect on online viewing.
 
 
\subsection{Horizontal Figures and Tables}
Figures and tables may be formatted horizontally
(a.k.a.\ landscape) as long as their captions appear
horizontal also.  \LaTeX\ will format landscape material for you.

Include the \texttt{rotating} package 
\begin{demo}
\\usepackage[figuresright]\{rotating\}
\end{demo}
and read the documentation that comes with the package. 

Figure~\ref{sideways} is an example of how a landscape
table might be formatted. 

\begin{figure}[t]
\uwsinglespace
\begin{verbatim}
     \begin{sidewaystable}
         ...
         \caption{ . . . }
     \end{sidewaystable}
\end{verbatim}
\caption[Generating a landscape table]{This text would create a
  landscape table with caption.}
\label{sideways}
\end{figure}
 


\subsection{Figure and Table Captions}
Most captions are formatted with the \verb"\caption" macro as described 
by Lamport\cite[sec. C.9]{Lbook}. 
The uwthesis class extends this macro to allow
continued figures and tables, and to provide multiple figures and
tables with the same number, e.g., 3.1a, 3.1b, etc.
 
To format the caption for the first part of
a figure or table that cannot fit
onto a single page use the standard form:
\begin{demo}
\\caption[\textit{toc}]\{\textit{text}\}
\end{demo}
To format the caption for the subsequent parts of 
the figure or table 
use this caption:
\begin{demo}
\\caption(-)\{(continued)\}
\end{demo}
It will keep the same number and the text of the caption will be 
{\em(continued)}.

To format the caption for the first part of
a multi-part figure or table
use the format:
\begin{demo}
\\caption(a)[\textit{toc}]\{\textit{text}\}
\end{demo}
The figure or table will be lettered (with `a') as well as numbered.
To format the caption for the subsequent parts of 
the multi-part figure or table
use the format:
\begin{demo}
\\caption(\textit{x})\{\textit{text}\}
\end{demo}
where {\em x} is {\tt b}, {\tt c}, \ldots.
The parts will be lettered (with `b', `c', \ldots).

If you want a normal caption, but don't want a ToC entry:
\begin{demo}
\\caption()\{\textit{text}\}
\end{demo}
Note that the caption number will increment.  You would normally use this 
only to leave an entire chapter's captions off the ToC.


\subsection{Line spacing}

Normally line spacing will come out like it should. However, the 
ProQuest style allows single spacing in certain situations:
figure content, some lists, and etc.
Use \verb"\uwsinglespace" to switch to single spacing within
a \verb"\begin{}" and \verb"\end{}" block.
The code examples in this document does this. 

\section{The Preliminary Pages}
 
These are easy to format only because they are relatively invariant
among theses.  Therefore the difficulties have already been encountered
and overcome by \LaTeX\ and the thesis document classes.

Start with the definitions that describe your thesis.
This sample thesis was printed with the parameters:

\begin{demo}
\\Title\{The Suitability of the \\LaTeX\\ Text Formatter\\\\
   for Thesis Preparation by Technical and\\\\
   Non-technical Degree Candidates\}
\\Author\{Jim Fox\}
\\Program\{IT Infrastructure\}
\\Year\{2012\}

\\Chair\{Name of Chairperson\}\{title\}\{Chair's department\}
\\Signature\{First committee member\}
\\Signature\{Next committee member\}
\\Signature\{etc\}

\end{demo}
 
Use two or more \verb"\Chair" lines if you have co-chairs.
 
\subsection{Copyright page}
Print the copyright page with \verb"\copyrightpage".

\subsection{Title page}
Print the title page with \verb"\titlepage".
The title page of this thesis was printed with%
 
\begin{demo}
\\titlepage
\end{demo}
 
You may change default text on the title page with these
macros.  You will have to redefine \verb"\Degreetext", for instance,
if you're writing a Master's thesis instead of a dissertation.\footnote{If you use these they can
be included with the other information before \\copyrightpage".}

\begin{list}{}{\itemindent\parindent\itemsep0pt
   \def\makelabel#1{\texttt{\char`\\#1}\hfill}\uwsinglespace}
\item[Degree\char`\{{\it degree name}\char`\}]
   defaults to ``Doctor of Philosophy''
\item[School\char`\{{\it school name}\char`\}] defaults to
``University of Washington''
\item[Degreetext\char`\{{\it degree text}\char`\}] defaults to
``A dissertation submitted \ldots''
\item[textofCommittee\char`\{{\it committee label}\char`\}] defaults to
``Reading Committee:''
\item[textofChair\char`\{{\it chair label}\char`\}] defaults to
``Chair of the Supervisory Committee:''
\end{list}

These definitions must appear \underline{before} the \verb"\titlepage" command.

 
\subsection{Abstract}
Print the
abstract with \verb"\abstract".
It has one argument, which is the text of the abstract.
All the names have already been defined.
The abstract of this thesis was printed with
 
\begin{demo}
\\abstract\{This sample . . . `real' dissertation.\}
\end{demo}
 
 
\subsection{Tables of contents}
Use the standard \LaTeX\ commands to format these items.
 
 
\subsection{Acknowledgments}
Use the \verb"\acknowledgments" macro to format the acknowledgments page.
It has one argument, which is the text of the acknowledgment.
The acknowledgments of this thesis was printed with
 
\begin{demo}
\\acknowledgments\{The author wishes . . . \{\\it il miglior fabbro\}.\\par\}\}
\end{demo}
 
 
\subsection{Dedication}
Use the \verb"\dedication" macro to format the dedication page.
It has one argument, which is the text of the dedication.
 
\subsection{Vita}
Use the \verb"\vita" macro to format the curriculum vitae.
It has one argument, which chronicles your life's accomplishments.

Note that the Vita is not really a preliminary page.
It appears at the end of your thesis, just after the appendices.
 
 
%%  
%% \section{Customization of the Macros}
%%  
%% Simple customization, including 
%% alteration of default parameters,  changes to dimensions,
%% paragraph indentation, and margins, are not too difficult.
%% You have the choice of modifying the class file ({\tt uwthesis.cls})
%% or loading
%% one or more personal style files to customize your thesis.
%% The latter is usually most convenient, since you do not need
%% to edit the large and complicated class file.
%% 
 


% ========== Chapter 4
 
\chapter{Running \LaTeX\\
  ({\it and printing if you must})}
 
 
From a given source \TeX\ will produce exactly the same document
on all computers and, if needed, on all printers.  {\it Exactly the same}
means that the various spacings, line and page breaks, and
even hyphenations will occur at the same places.

How you edit your text files and run \LaTeX\ varies
from system to system and depends on your personal preference.

\section{Running}

The author is woefully out of his depth where 
\TeX\ on Windows is concerned.  Google would be his resource.
On a linux system he types

\begin{demo}
\$\ pdflatex uwthesis
\end{demo}

and it generally works.

 
\section{Printing}
 
All implementations of \TeX\ provide the option of {\bf pdf} output,
which is all the Graduate School requires.  Even if you intend to
print a copy of your thesis create a 
{\tt pdf}.  It will print most anywhere.

\printendnotes

%
% ==========   Bibliography
%
\nocite{*}   % include everything in the uwthesis.bib file
\bibliographystyle{plain}
\bibliography{uwthesis}
%
% ==========   Appendices
%
\appendix
\raggedbottom\sloppy
 
% ========== Appendix A
 
\chapter{Where to find the files}
 
The uwthesis class file, {\tt uwthesis.cls}, contains the parameter settings,
macro definitions, and other \TeX nical commands which
allow \LaTeX\ to format a thesis.  
The source to
the document you are reading, {\tt uwthesis.tex},
contains many formatting examples
which you may find useful.
The bibliography database, {\tt uwthesis.bib}, contains instructions
to BibTeX to create and format the bibliography.
You can find the latest of these files on:

\begin{itemize}
\item My page.
\begin{description}
\item[] \verb%https://staff.washington.edu/fox/tex/thesis.shtml%
\end{description}

\item CTAN
\begin{description}
\item[]  \verb%http://tug.ctan.org/tex-archive/macros/latex/contrib/uwthesis/%
\item[]  (not always as up-to-date as my site)
\end{description}

\end{itemize}

\vita{Jim Fox is a Software Engineer with IT Infrastructure Division at the University of Washington.
His duties do not include maintaining this package.  That is rather
an avocation which he enjoys as time and circumstance allow.

He welcomes your comments to {\tt fox@uw.edu}.
}


\end{document}
