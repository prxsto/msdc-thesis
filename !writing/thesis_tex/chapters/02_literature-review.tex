\chapter{Literature review}

\section{The housing reality}

In the United States, the traditional method to grow one’s financial worth is through the process of becoming a homeowner. To many nowadays, this is simply impossible. Those who have been lucky to be able to afford a mortgage may even attempt to purchase a second property to rent as landlord. In the grand scheme of things they are still likely just trying to get by and provide for their family, but it is at the cost of those who are stuck in the cycle of renting. Renting has become the new culture du jour for many Americans. Between 1960 and 2017, the median house price in Seattle has increased 286\%, while the median household income has only increased 59\%. This disparity has created a paradigm in which the average income earner cannot afford to take out a mortgage for even a historically modest "starter" home \cite{tekinTimelineAffordabilityHow2021}.

In the United States there are 44 million renters, and of these more than 47\% are rent burdened \cite{kontokostaEnergyCostBurdens2020a}. The term ‘rent burdened’ is defined by the United States Department of Housing and Urban Development (HUD) as a household that spends more than 30\% of their gross annual income on rent. The US is also one of few countries in the developed world which does not have any successful form of public housing in effect due to traditionally neoliberal housing policy. Countries such as Austria boast more affordable, and higher-quality public housing offerings than the average rental in the United States, with the public costs coming primarily from taxes upon large corporations \cite{forrestViennaAffordableHousinga}.

The behaviors that govern this issue are systemic, but also unconscious for the most part on the side of the homeowners. This same group makes up the bulk of those attending municipal meetings regarding land use and urban expansion to fight for their own interests. This is not a desire to price out others from purchasing their first home, but rather the desire to increase their home valuation. Historically however, some have actively used this exclusionary strategy to deter black and brown people from integrating into suburbs for example. On the other hand, there are housing activists fighting for a more just and affordable reality for city-dwellers. Driven primarily by the philosophy of new urbanism, this discourse primarily relates to ideas such as rent control, public housing options, expanding multifamily residential zoning, and the creation of more green spaces \cite{10.2307/24392672}.

Redlining, or the increase in interest rates due to a perceived ‘risk’ to issue mortgages in varying neighborhoods based on ethnic makeup or crime rates has legally been eliminated, but other issues continue to plague the US housing market. The twin concepts of filtering and gentrification, while not systemic, are unconscious systems that impact housing affordability. Filtering is defined as when older buildings age, the initial wealthy occupants move to newer, more expensive housing and this allows those with lower incomes to move in \cite{hertzWhatFilteringCan}. Gentrification, on the other hand, has the opposite effect. Boutiques, cafes, and other non-essential shops opening in historic, filtered districts spike the housing prices back up or above the original prices, considering inflation. Many of these older dwellings are retrofitted into luxury apartments far out of reach of the existing residents. In turn, this causes many of those who have lived there for decades to be displaced to nearby neighborhoods once they can no longer afford the increased rental prices. 
 
While affordable housing is subsidized through various methods, either publicly or privately, homeowners are also subsidized in ways not typically discussed. Homeowners are given mortgage interest tax deductions, leading towards mortgages often being more affordable than renting. However, many current renters are unable to secure mortgage loans due to credit history or negative marks due to landlord issues or past evictions. In addition to this, in the beginning of 2021 began a new force generating the impossibility of purchasing a home for millennials and generation Z. Private equity firms such as BlackRock are purchasing massive amounts of starter homes to convert to rentals. In just the first quarter of this year, 15\% of all home sales were to private equity like BlackRock, some paying upwards of \\ 20\% - 50\% over asking price in cash in an effort to outcompete prospective buyers, while paying nearly anything in interest \cite{botellaInvestmentFirmsAren}. 

\section{Impact of energy use on affordability}

\textbf{[Very WIP - this is actually an outline rendered incorrectly]}

Climate change numbers accounted for within architecture/construction industries
one third of all carbon emissions stem from architecture and related industries @vectorfieldbasedsupportbuildingenergyconsumption
ECB
Filtering effect on ECB: 
However, by this time the building is usually in disrepair and far behind in terms of energy efficiency, further leading to a burden on the occupant \cite{kontokostaEnergyCostBurdens2020a}. 
Energy use not only affects the cost for potential residents, but also may affect the decision whether to construct or not in the first place. When operating costs from energy use are low, the owner of the ADU may pass those savings onto the resident. The majority of ADU owners in an Oregon survey were found to have paid for the cost of construction upfront using cash@ADUreport.
ADUs will need to be more energy efficient than the average dwelling
the proportion of income spent on energy costs falls disproportionately onto lowest income households in the united states [@kontokostaEnergyCostBurdens2020a]
due to [[Filtering]], low income families tend to inhabit units which are innately energy inefficient
those with <80\% median area income spend median [[Energy Cost Burden]] of 7.2\%-25\% [@kontokostaEnergyCostBurdens2020a]
affects latino and black families disproportionately
in seattle, energy costs tend to lead to a lower ECB on renters; however, still important to take into account
potential ADU owners also have stake in lowering energy costs of unit
lower utility costs means lower rent, further cutting costs and enticing renters

\section{Accessory dwelling units as a solution}

\textbf{[Very WIP - this is actually an outline rendered incorrectly]}

2.3.1 Existing ADU code in Seattle, WA (look at website and 597 article)
Allowed since 1994
Updated july 2019 policy: 2 adus per lot, off-street parking, owner-occupancy not required
ADU code around the country (ADUreport)
Potential changes for the better
Why ADUs (introduction outline)
Aging in place
15 minute city
	2.3.2 Why ADUs?
1929 was due to increased lot size, wave of bungalow-style home
bungalow style house had much worse use of land than previous single family or multifamily housing typologies
as average lot size increases, supply of land decreases, driving up housing prices on average
the first zoning codes came along with bungalow development
[[Urban Sprawl]] and necessity to own car in suburbs increased average energy use per american drastically
50\% of households in US owned vehicle by 1920 [@danielhertzImmaculateConceptionTheory15]
overall cost of living increased 2x, yet housing costs increased 5.5x [@danielhertzImmaculateConceptionTheory15]
politicians at local level have greatest power to affect change
majority of constituents in city politics tend to be homeowners [@10.2307/24392672]
wield political power to defend land prices, lower taxes
at the same time, development of zoning codes explicitly done to protect property values- history does repeat itself
desirability directly proportional to property value
therefore, residential segregation/redlining implemented
covenants, zoning, violence all used
'homevoter hypothesis' - [[William Fischel]]
price of available housing supply due to land use regulations and therefore, limited supply of developable land
best way to lower housing costs and increase density is to construct multifamily residential buildings
increased density lowers property values for nearby homeowners, making it increasingly difficult to build multifamily residential even in the correct zoning *(find citation for this)*
as mentioned previously, [[ADU]]s offer a way to increase housing density in residential neighborhoods with less pushback
reduces necessity to own car (as long as needs are met within 15 minute walking distance) [[15 Minute City]], reduces urban sprawl
does not require full revision of land use code
increases property value for owner
enables affordable inclusion of those races and classes formerly restricted
enables older citizens to remain independent
[[Aging in Place]]
ADUs remedial option, offers quick relief but not whole solution (find more sources on this)

 
\section{Energy modeling}

\textbf{[Very WIP]}

\begin{itemize}
 	\item Accuracy
	\item Speed 
	\item EnergyPlus
	\item State of the industry today/how it’s used
	\item Cove.tool: proprietary software that allows for machine-learning based energy analysis. 
	Downside is it is proprietary- talk to benefits of free and open source software, energyplus included
	\item further research in this area
\end{itemize}


\section{Surrogate modeling via machine learning}

\textbf{[Very WIP]}

\begin{itemize}
	\item Caitlin mueller: Computational Exploration of the Structural Design Space and \\ An Integrated Computational Approach for Creative Conceptual Structural Design
	\item Type of ML used, errors used
	\item deep learning algorithms used for EUI prediction - why use over standard ML?
	\item can a machine design? - nigel cross / slight background
	\item Machine learning for architectural design: Practices and infrastructure
	\item Further perspectives: data in design practice
\end{itemize}

