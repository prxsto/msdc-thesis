\chapter {Introduction}

\section{Context}

The unprecedented housing crisis in which the United States is currently facing requires both updated, progressive policy and novel design ideas. Stabilizing property values is socially beneficial to all residents \cite{10.2307/24392672}. Hence, a twin focus on equity and sustainability is required for an effective solution. This predicament the housing market finds itself in can largely be traced to zoning regulation. Single family housing zones across the country, as they currently exist, restrict the density necessary to house all citizens at just prices. Resultantly, rising housing costs, whether rental or mortgage, stem from a decreased supply of housing. For many, housing costs account for a greater share of their income, restricting their ability to afford basic necessities and decreasing overall quality of life. Due to the commodification of the housing market, increasing amounts of residents are becoming homeless and unable to find a foothold back into permanent housing \cite{routhierStateHomeless20212021}. As the realities of climate change begin to set in, the method employed to ease the housing crisis must also address sustainability. Further, people of color and those with lower levels of income are disproportionately affected by skyrocketing housing prices, and are also those most at risk of the enlarged rate of natural disasters due to our changing climate. Low income residents currently encounter asymmetric energy costs when compared to wealthier citizens, widening the disparity in the cost of living \cite{kontokostaEnergyCostBurdens2020a}. 

Traditional means of increasing housing stock is by increasing density and constructing multifamily housing structures. However, since the 1920’s, the trend has been larger plots of land per owner, leading to urban sprawl \cite{hertzImmaculateConceptionTheory15}. Proposed changes to land use policy is frequently met with pushback from homeowners due to lesser demand equaling a reduced property value. Ideally, single-family residential zones close to higher-density zoning would gradually be upzoned, or converted into multifamily residential or mixed, multifamily residential and commercial. As this is not currently feasible, cities around the country have begun to experiment and implement new regulations allowing for the construction of accessory dwelling units (ADUs). Accessory dwelling units are separate structures residing within existing single-family residential lots, typically used as rental housing, office space, or for housing an aging loved one close to home. ADUs fall into two sub-categories: attached and detached, with the former being separate, freestanding structures, while the latter is an expansion onto the existing structure. This thesis will be focusing solely on detached accessory dwelling units, or DADUs.

Due to their size, ADUs allow individuals to downsize the amount of space they occupy, reducing energy use and carbon footprint per capita. ADUs being a comparatively affordable typology of investment, combined with their intrinsic ability to increase property value, has led the way for new policy to allow for their construction \cite{chappleJumpstartingMarketAccessory2017}. Beyond restrictive zoning and permitting regulations, price contributes to the rate of accessory dwelling unit construction. In fact, recent 2019 survey data collected by the City of Seattle shows that homeowners desire an increased focus on sustainability and cost. Discovering an effective method for a reduction in energy use will increase construction rate of ADUs, benefitting both owners and occupants, while addressing climate change and housing injustices. It is of utmost importance for architects, policymakers, and residents to recognize that energy use and housing availability are not solely problems of climate or equity, respectively.\\

Traditional methods for evaluating energy performance of a proposed structure involve what is referred to as the modeling approach. This method requires an accurate 3D model to be constructed with specific materials applied, along with geographic information. The model is then simulated over the course of one calendar year to evaluate energy performance metrics, including energy use intensity (EUI). EUI is defined as the total energy consumed by a building over the course of one year, divided by gross square footage of the building (\(\frac{kW}{m^2\cdot yr}\)). This metric is helpful to compare energy use between buildings of varying sizes and typologies. EnergyPlus, an open-source building energy simulation program, is the most widely-used framework for this method in the United States. While precise, the modeling approach is time consuming on the scale of an individual design project. In contrast, the statistical approach is gaining popularity. These methods for calculating building energy performance rely on statistical inference using specified design inputs. This thesis investigates the use of surrogate modeling as an improved method to derive EUI. Surrogate modeling uses statistics and computation to estimate values to a high degree of accuracy. Using machine learning models to calculate EUI saves time for designers, but requires extra upfront resources to train.

\section{Previous work}

Efforts to increase the production of ADUs across Seattle have varied in goal and scope.  Beginning in 1994, Seattle began to allow ADUs only in single-family zones due to requirements in the Washington Housing Policy Act \cite{levyADUsPoliciesRacial2019}. However, the limitation at the time was that only AADUs were allowed. Following up on this, Seattle expanded the ADU rules to allow for DADUs in 2010. As only 50 out of 100,000 eligible properties constructed ADUs since this change, Seattle began exploring policies to make accessory dwelling units more accessible in 2014. In the process, the city conducted a Racial Equity Toolkit (RET) in 2018 to determine if proposed policy changes could reduce racial disparities in housing \cite{welchAccessoryDwellingUnits2021}. This report discovered that removing barriers to ADUs was beneficial to both affordability and displacement through increasing the choice involved in housing. Yet, it was found that removal of regulatory barriers was inadequate alone, resulting in the expansion of the Home Repair Loan Program. This modification allowed for greater access of funds to low income families to construct additional housing for either family or community members on existing properties \cite{welchAccessoryDwellingUnits2021}.

In 2019, an initiative to inform potential ADU owners named ADUniverse was created as a joint project between Seattle Office of Planning and Community Development (OPCD) and the Data Science for Social Good Program at the University of Washington \cite{mohlerADUniverseToolEScience2019}. ADUniverse’s goal was to understand where ADUs have historically been built in Seattle, and to identify potential issues with eligible lots, while attempting to estimate costs associated. As part of this project, a survey was conducted later in September of 2019 to begin to understand the design criteria most important to potential owners. Results showed that “low cost” was the criteria in first place with 48\% of those surveyed responding “very important”, followed by ‘green building’ at 46\%. Other more-specific priorities suggested include: longer-term environmental costs, site specific considerations, and predictability in both construction and cost \cite{PreapprovedPlansAccessory2019}. The results of this survey were then used to inform design submissions from firms, with ten of these being selected later by the City of Seattle to be featured on their website as pre-approved plans.

While these pre-approved designs are from many renowned firms who do indeed focus on sustainability, they will fall short on energy efficiency. Due to the site-agnostic nature of pre-approved plans, unique characteristics of each site will have influences on their effectiveness: shading from surrounding objects and structures, orientation to existing buildings, solar gain, etc. On the other hand, being pre-approved, many headaches regarding permitting and construction can be bypassed. To fully remedy this situation a modular approach would be recommended. In the meantime, the collective improvements to access and modifications to ADU rules has proven successful.

Following the new City of Seattle rules implemented in 2019 regarding ADUs and the collection of 10 pre-approved DADU designs in 2020 as part of ADUniverse, ADU production has increased immensely. As this comes during a global pandemic, the values would likely be larger under normal circumstances. According to OPCD, in 2020 the number of AADU and DADU permits increased 53\% and 112\% respectively, over similar numbers from 2019 \cite{welchAccessoryDwellingUnits2021}. With this increase in production comes a need to continue to maintain this boost in production while simultaneously optimizing the energy use of these newly permitted units.



\section{Research questions}

As ADU production ramps up dramatically in Seattle, and across the country, there must be an emphasis on the sustainability of each new construction. 

\bf{Main research questions}
\begin{itemize}
	\item Can a predictive design tool be leveraged increase production of DADUs in Seattle while reducing energy and carbon intensity of new units?
	\item Can design results generated by the tool verify city planning policy initiatives?
\end{itemize}

\bf{Secondary questions}
\begin{itemize}
	\item What effects on EUI does sharing walls (removing rear setbacks) and introducing a floor-area ratio benefit for retaining existing structures have?
	\item How does the energy performance of the 10 pre-approved DADU designs compare to a site-specific DADU design?
	\item What is the effect of the design tool empowering potential DADU owners to see energy quantities beforehand?
	\item Which design constraints impact DADU energy use the greatest?
\end{itemize}


\subsection{Aims and objectives}

Providing designers with real-time energy and carbon consequences of design decisions would greatly increase sustainability efforts across the industry. 

tool useful within existing code even if results dont push foward policy
 
1. provide designers with realtime energy and therefore carbon consequences of design decisions
2. how results can prove policy

\section{Research methods and outline}

%TODO 
\textbf{[WIP] Write miniature methods section and detail the outline of the thesis}
